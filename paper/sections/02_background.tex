\section{Institutional Background and Literature}\label{sec:background}

\subsection{The Swiss Cultural Divide}

Switzerland's linguistic border runs roughly north--south through the cantons of Bern, Fribourg, and Valais, separating a German-speaking majority (about 63\% of the population) from a French-speaking minority (23\%). The border has been remarkably stable since the early modern period; it does not coincide with cantonal boundaries, administrative districts, or major geographic barriers, making it a credible source of exogenous variation in cultural exposure \citep{bruegger2009}.

A growing body of evidence documents systematic preference differences along this border. \citet{eugster2017} show that German-speaking communities are 10 percentage points less supportive of redistributive social insurance, an effect that survives the inclusion of municipality fixed effects and income controls. \citet{steinhauer2018} find that female labor-force participation drops discontinuously at the R\"{o}stigraben, consistent with more traditional gender-role norms in the German-speaking heartland. \citet{lalive2009} demonstrate that the structure of parental leave policy interacts with these cultural norms in shaping fertility and labor-supply decisions. Taken together, these studies suggest that the linguistic border captures something deeper than language: a cluster of attitudes toward effort, independence, and the appropriate balance between market and non-market time.

Critically for our purposes, all of these preference differences exist within a unified institutional framework. Federal labor law sets uniform standards for maximum working hours (45--50 hours per week depending on the sector), overtime compensation, and contract termination. Unemployment insurance is governed by a single federal statute with identical replacement rates and benefit durations. Tax schedules are set at the federal and cantonal levels, but the cantonal variation does not align neatly with the language border. This institutional uniformity allows us to attribute differential trends in overwork to cultural factors rather than to regulatory differences.

\subsection{The Economics of Working-Time Preferences}

The neoclassical labor-supply model predicts that hours are determined by the intersection of wage rates and preferences for leisure. In a frictionless market, actual hours equal desired hours. Reality departs from this benchmark in two well-documented ways. First, employers face coordination costs and fixed costs per worker that incentivize the bundling of hours into standardized schedules, creating ``hours constraints'' \citep{hamermesh1999}. Workers may be unable to choose their preferred hours and instead select from a discrete set of employer-offered packages. Second, social norms shape the reference point against which hours are evaluated. A 42-hour work week is ``normal'' in Zurich but may be perceived as excessive in Geneva.

The concept of ``overwork''---the gap between actual and desired hours---has been studied extensively in the European context. \citet{bell2012} show that about 40\% of British workers report a mismatch between actual and desired hours, with the majority preferring fewer hours at a proportionally reduced wage. \citet{otterbach2016} exploit German panel data to show that hours mismatches reduce life satisfaction, with the effect concentrated among workers who report being overworked rather than underworked. \citet{green2012} demonstrate that work intensity and long hours reduce subjective well-being, with effects that vary across skill groups.

\subsection{Cultural Norms and the Cost of Effort}

The withdrawal of discretionary effort---doing what is contractually required and no more---can be understood as a shift in labor supply at the intensive margin. In a standard framework, workers supply effort until the marginal utility of income equals the marginal disutility of labor. However, this "disutility" is not a biological constant; it is socially constructed. \citet{alesina2005} argue that social norms play a crucial role in determining the equilibrium level of hours worked, distinguishing between cultures that prioritize income generation and those that prioritize leisure.

If Germanic work-ethic norms create stronger social-compliance pressure to demonstrate dedication through long hours \citep{bell2012}, then "overwork" may be perceived as a virtuous signal or a necessary duty. In contrast, in French-speaking regions, where cultural baselines may place greater value on the protection of private time \citep{bruegger2009}, the same level of overwork may be experienced as a costly violation of the social contract. This suggests that the "disutility" of an extra hour of work is higher in the French-speaking region not because the work is harder, but because it conflicts more sharply with the prevailing cultural schema of a good life.

Our contribution is to test this hypothesis using within-country variation. Rather than comparing countries with different labor laws, we compare individuals who face the same legal and economic incentives but differ in the cultural lens through which they evaluate the effort--reward bargain.
