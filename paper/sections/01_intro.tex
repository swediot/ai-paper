\section{Introduction}

Modern labor markets are characterized by a persistent tension between contractual obligations and the implicit expectation of discretionary effort. While economic theory typically models labor supply as a tradeoff between leisure and income, the psychological cost of "overwork"---the gap between actual and contractual hours---may depend heavily on the cultural scripts through which workers interpret their professional obligations. When does an extra hour of work feel like a fair exchange, and when does it feel like a violation of the boundary between public and private life?

We argue that the psychological cost of overwork is, to a significant degree, culturally determined. Testing this claim requires a setting in which individuals share identical institutional constraints but differ in deep-seated cultural norms. Switzerland provides precisely such a natural experiment. French-speaking and German-speaking workers operate under the same federal labor code, face the same tax schedules, and participate in the same macroeconomic environment, yet they inherit strikingly different historical attitudes toward work, leisure, and the boundary between professional and personal life \citep{bruegger2009, eugster2017}. The linguistic border that separates these communities---colloquially known as the \emph{R\"{o}stigraben}---allows us to isolate the effect of cultural norms on the welfare costs of labor supply.

Using 25 waves of the Swiss Household Panel (SHP, 1999--2023), we construct a longitudinal dataset of employed individuals in German- and French-speaking cantons. We define "overwork" as the gap between actual weekly hours worked and contractual hours per week. Exploiting within-person variation, we estimate how each additional hour of overwork differentially affects French-speaking workers across a battery of burnout outcomes---post-work exhaustion, work-life interference, and difficulty disconnecting---as well as domain-specific satisfaction measures.

Three main findings emerge. First, overwork is significantly more psychologically costly for French-speaking workers: each additional hour beyond the contract increases post-work exhaustion by 70\%, work-life interference by 112\%, and difficulty disconnecting by 34\% in French-speaking regions (all $p < 0.02$). Second, overwork also reduces satisfaction with free time ($\hat{\beta}_3 = -0.024$, $p < 0.001$) and work conditions ($\hat{\beta}_3 = -0.007$, $p = 0.01$) more for French-speaking workers, while global life and job satisfaction show no significant cultural moderation---indicating domain-specific rather than diffuse effects. Third, these effects are persistent across demographic groups, though managers face notably higher penalties.

We contribute to the labor economics literature on hours constraints and preference heterogeneity \citep{goldin2014, mas2017}. While previous research has documented that workers value schedule flexibility and are willing to pay for it, we show that the *cost* of deviating from a standard schedule is not uniform but culturally modulated. Our results suggest that identical labor-supply behaviors generate widely divergent welfare effects depending on the cultural schema through which they are experienced.

The remainder of the paper proceeds as follows. Section~\ref{sec:background} reviews the institutional context and the relevant literatures. Section~\ref{sec:data} describes the data and our econometric approach. Section~\ref{sec:results} presents the findings. Section~\ref{sec:conclusion} concludes.
