\section{Conclusion}\label{sec:conclusion}

This paper has used Switzerland's linguistic border---the R\"{o}stigraben---to identify the role of cultural norms in shaping the \emph{psychological cost} of overwork. Exploiting 25 waves of the Swiss Household Panel and the fact that French- and German-speaking workers share identical institutional constraints, we show that overwork is sharply more costly for French-speaking workers: they experience significantly more exhaustion, work-life interference, and difficulty disconnecting per hour of overwork, as well as steeper declines in free-time and work-conditions satisfaction. All key burnout interactions are significant at the 1\% level or better.

These findings challenge the view that the relationship between work effort and well-being is universal or purely determined by economic incentives. Our evidence suggests that deep-seated cultural schemas determine how workers evaluate the trade-off between income and personal time. A French-speaking worker in Lausanne and a German-speaking worker in Zurich may face the same labor market conditions, yet the cultural lens through which they interpret overwork differs meaningfully. For the former, overwork is a more costly violation of the boundary between professional and private life.

Two practical conclusions follow. First, employers across cultural contexts cannot assume that wage premia will be equally effective in compensating for long hours. In regions where cultural norms legitimize boundary-setting, financial incentives may be a poor substitute for organizational policies that respect contractual limits---especially since our evidence shows that overwork generates measurably higher exhaustion and work-life conflict in these regions. Second, workplace well-being programs should be culturally calibrated: the burnout channels through which overwork damages workers differ across cultural contexts. In French-speaking regions, interventions that protect the work-life boundary (e.g., right-to-disconnect policies, strict overtime limits) may be more effective than general stress management programs.

Our study faces four limitations. First, the hours-gap measure is based on self-reported actual and contractual hours, both of which are subject to recall error and social-desirability bias. If French-speaking workers are more willing to admit to working fewer hours, our differential effect could partly reflect reporting differences rather than behavioral differences. Second, the SHP does not include direct measures of cultural attitudes toward work effort, forcing us to rely on interview language as a proxy for cultural exposure. Future work could exploit direct survey measures of work-ethic norms or experimental designs. Third, our individual fixed-effects strategy identifies effects from within-person variation over time. Selective attrition---if workers who withdraw effort also exit the panel---could bias our estimates. Fourth, we exclude Italian-speaking respondents due to their small sample size; whether similar cultural moderation effects operate along the Italian--German border remains an open question.

Despite these caveats, our central message is robust: culture modulates not just the behavior but the \emph{psychological experience} of overwork. The negotiation of working norms is not simply a matter of economics but a set of culturally inflected responses, with distinct burnout channels that vary systematically across the R\"{o}stigraben.
